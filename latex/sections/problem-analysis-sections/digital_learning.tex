\section{Digital Learning}\label{ch:digital_learning_ch}
\subsection{What is digital learning?}\label{ch:what_is_digital_learning}
\subsection{Visual learning and its properties}\label{ch:visual_learning_and_its_properties}
\subsection{Spatial intelligence and awareness}\label{ch:spatial_intelligence_and_awareness}

\subsection{The ability to visualize}\label{ch:the_ability_to_visualize}
\subsection{Visual and text representation}\label{ch:visual_and_text_representation}

\section{Opbygning - slet senere}\label{ch:opbygning}
Digital learning - Introduktion til kapitlet, hvorfor skal vi vide det her i forhold til problemet?

What is digital learning? - Hvad består begrebet digital learning af og hvad er de primære elementer i det? 
    Hvem bruger det og hvem gavner det? 
    Hvilke værktøjer benytter man til den digitale proces?

  Visual learning and its properties - Den digitale proces er visuelt repræsenteret
    kræver mindre interaktion, der står en del om forskellen på e-learning og digital learning somewhere
    fordele/ulemper ved visuel læring generelt (masser af keywords nedenfor), husk at skelne generelt og mennesket

  Spatial intelligence and awareness
    Lidt dybere ned, hvor mennesket er i fokus
    Spatial reasoning
    et billede siger et tusinde ord
    Fotografisk hukommelse
    Need to see it to believe it (doctor - patient relation)
    fordele/ulemper (keywords)
    Læringsstile og forskellige intellekter (spatial, verbal)
    1. Spatial visualizeres 2. object visualizers
    EARLY learning -> evt. bruges til projektforslag (ellers evt. skip)
  
  The ability to visualize
    Menneskets evne og/ellers computers evne til at formidle visuelt 
    A geoscientist mentally manipulates the movement of tectonic plates to see the process of earth formation.
    A neurosurgeon visualizes different brain areas to predict the outcome of a surgery.
    A civil engineer imagines how various forces may affect the design of a system.
    Architects and engineers use material of various shapes and sizes to create stable structures.
    A designer uses visual spatial reasoning concept to enhance the user experience of his product.
    An artist creates stunning visual arts.
    A gymnast uses spatial awareness to perform a sequence of movements with the human body.

  Visual and text representation
    Nemmere ved at repræsentere forskelle, medianer, yder punkter, statistikker, grafer osv. osv.
    Matematik -> geometri, patterns, arithmetic, number sense osv.

  Outro
    Hurtig konklusion og oplæg til næste kapitel
    

Keywords/sentences regarding digital learning: 
  Pros -> Learn from anywhere, anytime, learn in your own pace, revisit materials and redo lectures, track progress, gamifiction, personalized learning, supports blended learning, collaborative learning.

  Cons -> Limited teacher & student interaction, risk of social isolation, cheating is more brutal to monitor, digitally challenged people has trouble, issues with quality, requires self-motivation and time management, practical component of learning suffers, limited to specific disciplines, lack of development of communication skills, prolonged screen exposure.