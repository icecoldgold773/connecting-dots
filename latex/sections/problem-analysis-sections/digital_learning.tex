\chapter{Digital Learning}\label{ch:digitalLearning}
%Digital learning - Introduktion til kapitlet, hvorfor skal vi vide det her i forhold til problemet?
\section{What is digital learning?}\label{ch:whatIsDigitalLearning}

Digital learning is the use of technology, such as computers, the internet etc. to give education and aid learning. It allows students to learn at their own pace, from any location, and at any time, making education more accessible and flexible. 
This approach to learning can be beneficial for people with different needs and requirements. Digital learning enables increased access to education and there are multiple factors that has to be taken into consideration, but the main ones can be summed down to these points\cite{gosa2023}:

\begin{itemize}
\item Pace:
	Digital learning offers a broader choice of instructional tools and materials than would be available in a traditional classroom. this can be any educational material that can be found on the internet.\cite{gosa2023}
\item Locatition:
	  Digital learning allows students to access educational content from any location, as long as they have internet connection. This means that students do not need to attend a physical school if they cannot do so because of geographical boundaries or mobility issues. Digital learning also enables students to learn while being on the go.\cite{gosa2023}
\item Flexibility:
	Digital learning can be accessed anytime, anywhere. This allows students to work on their education around their own schedules, whether they are working, or pursuing other interests. Students can also choose to learn at their most productive times--early in the morning or late at night--which can be  beneficial for those who does not have time for traditional school.\cite{gosa2023}
\end{itemize}

However Digital learning is not only beneficial for students but can also benefit companies and organisations in giving new skills to their staff\cite{SkillsDigitalLearning}

\subsection{Digital Learning in work places}\label{ch:digitalLearningWorkPlaces}

The increasing use of technology in Industry 4.0 is expected to bring about a range of impacts, including reduced labor costs, greater flexibility, and shorter delivery times. It also promises to automate dangerous tasks, promote productivity growth, and lead to higher quality products.
Additionally, it is expected to result in safer surgeries, improved quality of life for the elderly and people with disabilities, and the creation of new products and services. However, these changes will also bring about new challenges in terms of employment and education, as well as changes in the way companies and organizations are structured.
A survey conducted on this topic indicates a low level of positive correlation between perceived challenges faced by organizations and opportunities for new disruptive business and new trends of skills.
It also shows a negative relationship between perceived challenges and organizational digital transformation. Overcoming these negative perceptions is necessary to adopt new trends in skill development and capitalize on new opportunities.\cite{SkillsDigitalLearning}
\subsection{Learning styles}\label{ch:learningStyles}
Some of the most popular learning styles are based on the VARK theory\cite{vark} created by Fleming and Mills. Here the learning styles get sorted into four main categories:
\begin{itemize}
\item \textbf{V}isual Learners \\
The preffered type of learning is visual. Learning through graphs, videos, slides, photos, figures etc.
\item \textbf{A}uditory Learners\\
The preffered type of learning is auditory. Learning through Discussions, Podcasts, ideas, analogies etc.
\item \textbf{R}eading Learners\\
The preffered type of learning is reading. Learning through textbooks, manuals definitions etc.
\item \textbf{K}inethestic Learners \\
The prefered type of learning is by doing. Learning through experience, testing, touching etc.
\end{itemize}
%There also exists an alternative learning style -Multimodal- which is a mix of all the other learning styles.
Learning is cost efficient, so by having the capability to imagine and visualize different objects and pattern, most of us can gain a quicker understanding through visual learning even though it might not be your preffered type of learning.\cite{pracpsych2022}.

\subsection{Visual learning and its properties}\label{ch:visualLearningAndItsProperties}
Visual learning envolves using your visual senses to help recognize material better. Things such as graphs, maps, images, animations etc. makes the proces of learning and gathering information more straightforward compared to the traditional textbook. The benefits that come with visual learning depends from person to person, but visual learning has shown to help people retain information better.
According to research, when students interact with their material using visual aids, their attention, focus, and motivation improve, resulting in stronger critical thinking abilities and an overall knowledge of the subject studied. Visual learning is also low-cost, with most strategies needing little to no investment. Simple strategies such as generating PowerPoint presentations with notes, marking significant passages, drawings, and flowcharts in notes, and creating flashcards can help to create a personalized and efficient learning experience.
It has been demonstrated that visual learning improves memory recall. While it is not the only learning method that can be used to retain information, research have shown that it is one of the most successful. Furthermore, visual aids such as presentations with pictures and charts, colorful notes with doodling, and diverse hues can make lectures and revising sessions more enjoyable.
It may assist in transforming even the most uninteresting subjects into enjoyable lessons that students look forward to participating in.\cite{visuallearning}

The people we refer to as having a "photographic memory" is more than often spatial learners, they are able to remember faces, objects and what they read by a picture in their head. Here seeing is really learning and thus visualizing the related content to these individuals are important\cite{herd1997}.

\section{Spatial intelligence and awareness}\label{ch:spatialIntelligenceAndAwareness}
(Lohman 1996) states that spatial intelligence, or visuo-spatial ability, has been defined as ``the ability to generate, retain, retrieve, and transform well-structured visual images'\cite[p97]{tapsfield1996}, with that definition in mind this section will describe the importance of spatial intelligence in regards to visual learning and spatial reasoning.

Data within behavioral and cognitive neuroscience has shown a distinct individual difference in visual imagery. Neural correlation between the occipital lope, temporal lope and towards the frontal lope has shown better abilities to proces color and pictorial details in objects of different shapes and sizes. The correlation between the occipital lope and parietal lope towards the frontal lope has likewise shown better abilities to proces location, movement and relations between objects while also being able to transform and manipulate them.
These two different pathways are also refered to the dorsal ``where' and ventral ``what' streams\cite{harvard2022}.
We will look further into the two differences below:
%indsæt picture
\begin{itemize}
  \item \textbf{The ventral ``what' stream} \\
   Individuals can use imagery to construct and visualize vivid images of objects, this relates to the appearence of objects in terms of their pictorial properties. While reading a book, these individuals can construct precise and detailed mental images of scenes from the book. Other examples most of us can relate to are obscured image games shared by our friends, where individuals with strong object processing abilities easily can make the distinction\cite{harvard2022}.
  \item \textbf{The dorsal ``where' stream} \\
   Individuals can have strong spatial processing abilities, some examples could be that these individuals will have better ability while playing spatial games like tetris, folding origami or just playing with Lego. The commom divisor in these games is being able to see the schematic representation of things and events\cite{harvard2022}.
\end{itemize}

\section{The ability to visualize}\label{ch:theAbilityToVisualize}
We probably all have heard the saying ``a picture is worth a thousand words', this saying was originally invented by and advertising executive (Fred R. Barnard)\cite{phrases2022}, like in advertising the power of the retainability of the message is in focus. 

%Need to see it to believe it (doctor - patient relation)
%farveblinde, ordblinde

%Menneskets evne og/ellers computers evne til at formidle visuelt 
%A geoscientist mentally manipulates the movement of tectonic plates to see the process of earth formation.
%A neurosurgeon visualizes different brain areas to predict the outcome of a surgery.
%A civil engineer imagines how various forces may affect the design of a system.
%Architects and engineers use material of various shapes and sizes to create stable structures.
%A designer uses visual spatial reasoning concept to enhance the user experience of his product.
%An artist creates stunning visual arts.
%A gymnast uses spatial awareness to perform a sequence of movements with the human body.

There are many benefits to visual representation, some of these benefits will be reviewed in the next section.
\section{Visual and text representation}\label{ch:visualAndTextRepresentation}
%Nemmere ved at repræsentere forskelle, medianer, yder punkter, statistikker, grafer osv. osv.
%Matematik -> geometri, patterns, arithmetic, number sense osv.


%OUTRO
%Hurtig konklusion og oplæg til næste kapitel
%Kun nogle af os kan have nytte af dette
%Sætter nogle krav til visualiseringen -> oplæg til machine learning?
%EARLY learning -> evt. bruges til projektforslag (ellers evt. skip)



%What is digital learning? - Hvad består begrebet digital learning af og hvad er de primære elementer i det? 
    %Hvem bruger det og hvem gavner det? 
    %Hvilke værktøjer benytter man til den digitale proces?

  %Visual learning and its properties - Den digitale proces er visuelt repræsenteret
    %kræver mindre interaktion, der står en del om forskellen på e-learning og digital learning somewhere
    %fordele/ulemper ved visuel læring generelt (masser af keywords nedenfor), husk at skelne generelt og mennesket

%Keywords/sentences regarding digital learning: 
  %Pros -> Learn from anywhere, anytime, learn in your own pace, revisit materials and redo lectures, track progress, gamifiction, personalized learning, supports blended learning, collaborative learning.

  %Cons -> Limited teacher %& student interaction, risk of social isolation, cheating is more brutal to monitor, digitally challenged people has trouble, issues with quality, requires self-motivation and time management, practical component of learning suffers, limited to specific disciplines, lack of development of communication skills, prolonged screen exposure.