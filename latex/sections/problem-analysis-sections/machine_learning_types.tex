\chapter{Types of Machine Learning}\label{ch:ml-types}
Within machine learning, there is a couple of different approaches to solving a problem. In this section, some of these
approaches will be analyzed along with the use cases of the different techniques. All approaches to machine learning can
be split into the following three categories:
\begin{itemize}
    \item Supervised learning
    \item Unsupervised learning
    \item Reinforcement learning
\end{itemize}


\section{Supervised Learning}\label{ch:supervised-learning}
Supervised learning is a widely used Machine Learning technique where a machine learning model trained 
on a labeled dataset can be used to predict the label of unlabeled data. The model adjusts its weights 
according to the accuracy of the model, allowing it to learn over time. The accuracy gets calculated through
a loss function which measures how far the predicted value is from the expected output.\cite{ml_supervised_learning}

Supervised learning models can be split into two categories, regression, and classification. Regression is used when
the output is continuous, while classification is used when the output is a discrete value. An example of a problem
where regression is used is predicting the price of a house based on different features such as the size of the house,
the number of rooms, and the location. An example of a classification problem is predicting if a received email is spam. \cite{ml_reg_vs_class}


\section{Unsupervised Learning}\label{ch:unsupervised-learning}
Unsupervised learning is a machine learning technique where a model is trained on a dataset without labels. The model then
tries to classify underlying patterns in the data without prior knowledge of what the output should be. In unsupervised learning,
it is challenging to evaluate the accuracy of a model since there is nothing to compare the output to. \cite{ml_unsupervised_learning}

One of the common approaches to unsupervised learning is clustering. Clustering is a technique where points in a dataset are grouped 
based on their similarity. Clustering is useful to find hidden patterns in a dataset that could be used to predict the outcome
of a problem. \cite{ml_unsupervised_learning}


\section{Reinforcement Learning}\label{ch:reinforcement-learning}
The last approach to machine learning is reinforcement learning. In reinforcement learning, a model is trained to perform a task by
receiving positive or negative feedback. The model is trained by going through a trial and error process, where it is rewarded for doing
the task correctly and punished for doing it incorrectly. \cite{ml_reinforcement_learning} 